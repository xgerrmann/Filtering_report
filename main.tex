\documentclass[11pt, oneside,a4paper,fleqn]{report}   
\usepackage[english]{babel}	
\usepackage{geometry}                		
\geometry{letterpaper}                   	 
\usepackage{graphicx}					
\usepackage{amssymb}
\usepackage{makeidx}
\usepackage{amsmath}
\usepackage{mathtools}
\usepackage[T1]{fontenc}
\usepackage[table]{xcolor}
\usepackage{latexsym}
\usepackage{amsfonts}
\usepackage{wasysym}
\usepackage{lmodern}
\usepackage{gensymb}
\usepackage{fixltx2e}
\usepackage{float}
\usepackage{lipsum}
\usepackage{graphicx}
\usepackage{mdwlist}
\usepackage{hyperref}
\usepackage{epsfig}
\usepackage{titlesec}
\usepackage[scaled=.90]{helvet}
\usepackage{wrapfig}
\usepackage[toc,page]{appendix}
\usepackage{bm}
\usepackage{epstopdf}
\usepackage{caption}
\usepackage{subcaption}


\hoffset = 0pt
\voffset = 0pt
\marginparwidth = 0pt
\textwidth = 450pt
\textheight = 550pt

\titleformat{\chapter}[display]   
{\normalfont\huge\bfseries}{\chaptertitlename\ \thechapter}{5pt}{\vskip -10pt\raggedright}{\Huge}   
\titlespacing*{\chapter}{0pt}{-50pt}{10pt}
\titlespacing*{\section}{0pt}{10pt}{10pt}

\graphicspath{{Images/}}

\begin{document}
\begin{titlepage}

\newcommand{\HRule}{\rule{\linewidth}{0.5mm}} 

\center 
 

 
%----------------------------------------------------------------------------------------
%	HEADING SECTIONS
%----------------------------------------------------------------------------------------

\textsc{\LARGE Delft University of Technology}\\[1.5cm] 
\textsc{\Large Filtering \& Identification}\\[0.5cm]
\textsc{\large SC42025}\\[0.5cm]

%----------------------------------------------------------------------------------------
%	TITLE SECTION
%----------------------------------------------------------------------------------------

\HRule \\[0.4cm]
Practical Assignment 2016-2017\\ Turbulence modeling for adaptive optics{ \huge \bfseries }\\[0.4cm] 
\HRule \\[1.5cm]
 
%----------------------------------------------------------------------------------------
%	AUTHOR SECTION
%----------------------------------------------------------------------------------------

\begin{minipage}{0.4\textwidth}
\begin{flushleft} \large
\emph{Authors:}\\
\textsc{Xander Gerrmann} (4105907)\\
\textsc{Dirk Mutters} (4105907)
\end{flushleft}
\end{minipage}
~
\begin{minipage}{0.4\textwidth}
\begin{flushright} \large
\emph{} \\
 \textsc{} % Supervisor's Name
\end{flushright}
\end{minipage}\\[1cm]

%----------------------------------------------------------------------------------------
%	DATE SECTION
%----------------------------------------------------------------------------------------

{\large \today}\\[1cm]

%----------------------------------------------------------------------------------------
%	LOGO SECTION
%----------------------------------------------------------------------------------------

%\includegraphics[width=0.5\textwidth]{Frontpage}\\ 
 
%----------------------------------------------------------------------------------------

\vfill 

\end{titlepage}


\tableofcontents


%----------------------------------------------------------------------------------------
%  Static Wavefront Reconstruction
%----------------------------------------------------------------------------------------
\chapter{\bf Static Wavefront Reconstruction}







%----------------------------------------------------------------------------------------
%  Vector Auto-Regressive Model of Order 1
%----------------------------------------------------------------------------------------
\chapter{\bf Vector Auto-Regressive Model of Order 1}
\section*{Q 4.1}
\begin{align}
    \phi(k+1) = A\phi(k) + w(k)\\
    E[w(k)e(k)^T] = 0,& &E[w(k)\phi(k)^T] = 0\\
    C_\phi(0)=E[\phi(k)\phi(k)^T],& &C_\phi(1)=E[\phi(k+1)\phi(k)^T]
\end{align}

Relating $ C_\phi(0)$,  $C_\phi(1)$ and $A$, using the covariance information:
\begin{align}
    C_\phi(1)&=E[\phi(k+1)\phi(k)^T]\\
    &=E[\big(A\phi(k) + w(k)\big)\phi(k)^T]\\
    &=E[A\phi(k)\phi(k)^T] + E[w(k)\phi(k)^T]\\
    &=A \cdot E[\phi(k)\phi(k)^T]\\
    &=AC_\phi(0)
\end{align}
So $A=C_\phi(1)C_\phi(0)^{-1}$, $C_\phi(0)$ is invertible, this is checked using Matlab.

\section*{Q 4.2}
\begin{align}
    \phi(k+1) = A\phi(k) + w(k)\\
    w(k) \sim \mathcal{N}(0,C_w)
\end{align}

\begin{align}
    E[ \phi(k+1) \phi(k+1)^T]&=E[A\phi(k)(A\phi(k))^T + w(k)(A\phi(k))^T + A\phi(k)w(k)^T + w(k)w(k)^T]\\
    &=AA^T E[\phi(k)\phi(k)^T] + A\cdot E[w(k)\phi(k)^T] + A\cdot E[\phi(k)w(k)^T] + E[w(k)w(k)^T]    
\end{align}
since $w(k)$ is uncorrelated with the turbulent wavefront $\phi(k)$, $E[w(k)\phi(k)^T]=0$ and $E[\phi(k)w(k)^T]=0$, so:
\begin{align}
    E[ \phi(k+1) \phi(k+1)^T]=AA^T C_\phi(0)+C_w
\end{align}
and since $\phi(k)$ is assumed to be a wide-sense stationary signal:
\begin{align}
    E[ \phi(k+1) \phi(k+1)^T]=E[ \phi(k) \phi(k)^T]=C_\phi(0)=AA^T C_\phi(0)+C_w\\
    C_w=C\phi(0)-AA^TC_\phi(0)=(I-AA^T)C_\phi(0)
\end{align}

\section*{Q 4.3}
\begin{align*}
    s(k)=G\phi(k)+e(k),& &\phi_{DM}(k)=Hu(k-1)\\
    \epsilon(k)=\phi(k)-\phi_{DM}(k),& &\phi(k+1)=A\phi(k)+w(k)
\end{align*}
Using the equations given above, a state-space model can be formulated in which the state is equal to the closed-loop residual wavefront $\epsilon(k)$. 
\begin{align}
    \epsilon(k+1)&=\phi(k+1)-\phi_{DM}(k+1)=A\phi(k)+w(k)-Hu(k)\\
    &=A\Big(\epsilon(k)+Hu(k-1)\Big)-Hu(k)+w(k)\\
    &=A\epsilon(k)+ 
        \begin{bmatrix}
            AH & -H 
        \end{bmatrix}
        \begin{bmatrix}
            u(k-1)\\
            u(k)
        \end{bmatrix}
        + w(k)\\
    s(k)&=G\phi(k)+e(k)\\
    &=G\Big(\epsilon(k)+Hu(k-1)\Big)+e(k)\\
    &=G\epsilon(k)+
        \begin{bmatrix}
            GH & 0 
        \end{bmatrix}
        \begin{bmatrix}
            u(k-1)\\
            u(k)
        \end{bmatrix}
        + e(k)
\end{align}

\section*{Q 4.4}
According to section 5.7 in the textbook, concidering the time invariant system
\begin{align*}
    x(k+1)&=Ax(k)+Bu(k)+w(k)\\
    y(k)&=Cx(k)+v(k)
\end{align*}
the Kalman filter associated to this system is
\begin{align*}
    \hat{x}(k+1|k)&=A\hat{x}(k|k-1)+Bu(k)+Ke(k)\\
    y(k)&=C\hat{x}(k|k-1)+e(k)\\
    \text{with }e(k)&=C(x(k)-\hat{x}(k|k-1)+v(k).
\end{align*}

Given our state-space model
\begin{align*}
    e(k+1)&=A\epsilon(k)+ 
        \begin{bmatrix}
            AH & -H 
        \end{bmatrix}
        \begin{bmatrix}
            u(k-1)\\
            u(k)
        \end{bmatrix}
        + w(k)\\
    s(k)&=G\epsilon(k)+
        \begin{bmatrix}
            GH & 0 
        \end{bmatrix}
        \begin{bmatrix}
            u(k-1)\\
            u(k)
        \end{bmatrix}
        + e(k)
\end{align*}

the associated Kalman filter is

\begin{align*}
    \hat{\epsilon}(k+1|k)&=A\hat{\epsilon}(k|k-1)+
    \begin{bmatrix}
        AH & -H 
    \end{bmatrix}
    \begin{bmatrix}
        u(k-1)\\
        u(k)
    \end{bmatrix}
    +K\eta(k)\\
    s(k)&=G\hat{\epsilon}(k|k-1)+\eta(k),\\
    \text{with }\eta(k)&=G\Big(\epsilon(k)-\hat{\epsilon}(k|k-1)\Big)+e(k)
\end{align*}
%----------------------------------------------------------------------------------------
%  Subspace Identification
%----------------------------------------------------------------------------------------
\chapter{\bf Subspace Identification}





%----------------------------------------------------------------------------------------
%  Comparison
%----------------------------------------------------------------------------------------
\chapter{\bf Comparison}

\bibliography{bibliography} 
\bibliographystyle{ieeetr}
\end{document}